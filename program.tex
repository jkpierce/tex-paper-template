\section{RESEARCH PROGRAM:} \label{sec:program}

The specific research program and its three year timelime are now outlined. Each project builds progressively toward the desired outcomes, which are (1) a validated computational physics model of exchange and kinematic statistics using measurable channel parameters, and (2) 3D observations of kinematic and exchange statistics in steady and unsteady flows, in conjunction with the underlying fluid and granular properties. Ten specific projects are now introduced. Each project is outlined with a problem statement, proposed solution, and anticipated production. Solutions employ the methods from \S \ref{sec:methods}. 

\subsection{SPECTRAL SYNTHETIC TURBULENCE FOR SEDIMENT TRANSPORT:} 

Problem: 

There is scope for an adaption of the spectral synthetic turbulence literature, mostly developed for aerodynamics and heat transfer applications, into a form useful for channel flow and geomorphic applications. In particular, this is required for the proposed SST-DEM model.  Flow over a rough bed is anisotropic and heterogeneous, and exhibits coherent turbulent structures \citep{Adrian2007}. Therefore, the spectral turbulence generator employed by previous authors in the sediment transport literature \citep{Bialik2012,Nikora2001} cannot express the coherent structures of wall-bounded turbulence. 
\bigskip 

Solution: 

More sophisticated spectral methods have been developed to generate inhomogeneous and anisotropic turbulence \citep{Smirnov2001, Clark2011, Shur2014, Adler2018}, but these have not been adopted by sediment transport researchers. After a careful literature review, one of these approaches will be adopted for the development of synthetic channel flow turbulence. 
\bigskip 

Production:
 
There is scope for a short article comparing the synthetic generators previously utilized in sediment transport with the improved generators available in the literature. In particular, the quadrant analysis scheme presented in \cite{Ferreira2009} can be employed to compare the coherent structures manifested by the \cite{Nikora2001a} and \citep{Bialik2013} work with those generated by the \cite{Shur2014} generator, for example. This will motivate these generators as useful and computationally efficient alternatives to large eddy simulation in computational sediment transport studies, and contribute to the development of the SST-DEM model. 
\bigskip 



\subsection{BOUNDARIES OF DIFFUSIVE SCALING REGIONS:}

Problem: 

\cite{Nikora2001a} proposed three distinct diffusive ranges: local, intermediate, and global. This idea has been elaborated extensively \citep{Hassan2015,Furbish2017}, and these ranges have been simulated \citep{Martin2012, Zhang2012, Bialik2015, Fan2016} and extracted through meta-analysis of experiments conducted at different scales \citep{Zhang2012}. However, the boundaries (time and space) of the three scaling regions have not been directly connected to the mechanics of particle transport. All simulations in the literature compute these ranges with modified random walk models having no reference to specific bed collision and resting and burial processes. Meanwhile, theoretical explanations of the diffusion ranges have explained them in terms of the preeminence of specific processes within each range \citep{Hassan2015,Furbish2017}. There is a disconnect between simulation and theory.  
\bigskip 

Solution: 

An early version of the SST-DEM model will be developed to study the diffusion of tracer particles through the virtual coupled fluid-granular system. The mean squared displacement $\bra x^2 \ket \propto D_u t^\alpha$ of the tracer population will be tracked across a long timescale of near-threshold transport. If the simulation can be run long enough, tracer diffusion should show an evolution of the scale parameter $\alpha$ through all three diffusion regimes. All tracers will have positions, exposures, and velocities tracked through the simulation. Therefore the evolution of $\alpha$ can be related to movement and resting phases, and the relative importance of fluid-induced acceleration, collision-induced deceleration, resting, and burial can be quantified in relation to each diffusive range. 
 
\bigskip 

Production: 

An article will be written relating the evolution of $\alpha$ through its three ranges to the predominance of specific transport and residence processes within each of the three sets of scales. The scaling boundaries will be discriminated in time and space, and this will be related to the corresponding length and time scales on which processes predominate. 
\bigskip 




\subsection{3D LAGRANGIAN BEDLOAD STATISTICS IN STEADY FLOW (1): KINEMATICS}

Problem: 

Lagrangian ensemble exchange and kinematic statistics are the foundation of stochastic sediment transport models. Unfortunately, there are very few measurements of these statistics in the literature \citep{Ancey2006, Lajeunesse2010, Roseberry2012, Heyman2013, Fathel2015,Heyman2016}, and none of these measurements have resolved all three dimensions of motion simultaenously. The first 3D phenomenon to be investigated is the well documented lateral variation of exchange characteristics \citep{Nelson2010, Hodge2013, Venditti2017}. This has not been resolved in Lagrangian experiments. The second 3D phenomenon of interest is the covariance of each velocity component with travel time. These covariances set the form of the velocity distribution expected from entropy analysis \citep{Furbish2013, Furbish2016}. One one component of this covariance has been investigated so far, and in only one set of experiments \citep{Fathel2015}. 
\bigskip 

Solution:
 
The proposed analysis of the binocular vision Lagrangian bedload tracking experiments will resolve 



\subsection{A TURBLENT GRANULAR MODEL OF BEDLOAD ENSEMBLES:} \label{proj:SSTDEM}

Problem: 

As outlined in \S \ref{sec:intro}, a practical link from measurable channel parameters to the ensemble statistics required by stochastic transport models is sorely needed for applications in ecology and engineering. Existing approaches with realistic treatment of a turbulent fluid phase coupled to a mobile granular phase are too computationally demanding for ready deployment \citep[e.g.][]{Schmeeckle2014,Elghannay2017}. 
\bigskip 

Solution: 
Therefore, the a coupled SST-DEM model treating realistic turbulence and grain-grain interactions will be developed. The reduced computational complexity of SST relative to other approaches, in addition to locking out of the degrees of freedom of particles below the active layer, will facilitate the computation of long transport time-series, generating large ensembles of trajectories through entrainment, motion, and deposition. 
\bigskip 

Production: 

The first presentation of the model will be an article exhibiting its general capabilities by studying the ensemble kinematics of bedload transport. For uniform 5mm glass beads under flow conditions just above threshold, the forms of hop length, velocity, acceleration, and travel time distributions will be presented, along with the covariance of hop length and travel time. These will be compared to recent theoretical \citep{Furbish2012,Furbish2016} and experimental \citep{Lajeunesse2010, Roseberry2012, Ancey2014, Fathel2015, Heyman2016} reports of these quantities, motivating the SST-DEM model as a useful tool for subsequent studies of mechanical closure.   


\subsection{COMPUTATION OF THE ENSEMBLE KINEMATICS OF BEDLOAD TRANSPORT:}

The model developed in project 4 will be employed to study kinematic statistics for uniform spherical particles over the parameter space of turbulent properties. This work will elaborate on previous efforts \citep[e.g.][]{Bialik2010, Ancey2014, Fan2014} in this category, with much more realism in the turbulent and granular forces. 
\bigskip 

One quantity of interest is the form of the kinematic distributions: velocity distributions are probably exponential-like \citep{Roseberry2012, Furbish2013,Fan2014, Fathel2015, Furbish2016}, but they may be Gaussian \citep{Lajeunesse2010, Ancey2014}, and they may even transition between these two possibilities depending on location in the water column \citep{Heyman2016}. As few experiments have been performed, exploring a restricted range of conditions, the wide variation in mean and turbulent flow statistics accessible by the model may lend some closure to this discussion of the form of kinematic distributions.  

\subsection{GRANULAR JAMMING AND PARTICLE MOBILITY:}

\cite{Zimmermann2010} proposed that forces between grains are a neglected component of particle mobility considerations, and hypothesized that their incorporation into mobility analysis may explain the anomalously high particle entrainment thresholds often observed in small and steep mountain streams \citep{Prancevic2015}. They considered that particles in steep mountain streams jam across the channel width, so that force chains between the grains stabilize them to entrainment. This hypothesis has received some support from reduced complexity modeling \citep{Saletti2016, Saletti2016a}, but this approach did not actually resolve force chains, nor have they ever been observed. Notably, the model developed in project 4 accounts for all inter-granular forces, so that jamming can be considered.  
\bigskip 

Therefore, this project proposes artificially increasing the transverse forces imparted by the wall onto the granular bed at a locality, and recording the resultant transverse distribution of forces through the bed across the channel, in concert with the areal entrainment rate of surface particles at this locality. This is expected to confer fundamental support for the jamming hypothesis, and shape subsequent research on bedload transport in mountain streams. Jamming is fully expected to emerge: the only questions concern its effect on entrainment thresholds. Jamming is very well documented in granular matter which is not subjected to shear flow \citep{Corwin2005, Liu2010}, and there is no obvious physical argument as to why shear flow would remove this possibility.  


\subsection{SEDIMENT ENTRAINMENT BY IMPACT EJECTION:}

As discussed, one aspect expressed by bedload transport is wide fluctuations in its rate, even under apparently steady conditions. This aspect is attributed, in one set of studies, to the phenomenon of collective entrainment, whereby the presence of moving particles in a locality is correlated to an increased entrainment rate in that locality  \citep{Ancey2006,Ancey2008,Heyman2014, Ancey2014, Heyman2016}. Determining the specific processes behind collective entrainment is important. For instance, the entrainment of multiple grains may occur because one grain entrained in a conventional way, and the other grains in contact with it were contingent on it for stability, so they too were entrained. Alternatively, the entrainment of multiple grains could occur because a particle in motion collided with the bed, imparting momentum to a group of otherwise immobile particles, leading to their collective entrainment. 

According to McEwan and Willets (1991) who investigated aeolian saltation, the
moving grains may be considered as made up from two components: aerody- namically entrained grains and impact generated grains. Further, McEwan et al. (1999) developed this approach, adapting it to the fluvial conditions, suggesting that the number of uniform grains Ne entrained per unit area per unit time is equal to
\bigskip 

This study will decompose entrainment events into individual and collective categories. Each category will then be partitioned into processes. Did the particle(s) entrain in a conventional way (that is, through a loss of force balance as considered by \cite{Wiberg1987}, or did it entrain due to impact ejection? This partition would be difficult and subjective in an experiment, but it is straightforward in a Lagrangian simulation. This project will extract the relative importance of the impact ejection process and conventional flow-based entrainment for presentation.  
\bigskip 


\subsection{DEPOSITION STATISTICS, BED ROUGHNESS, AND TURBULENCE: }

Bed roughness is hypothesized to increase the probability of deposition upon the collision of moving grains with the bed. Since the frequency of bed contant is encapsulated in kinematic statistics of bed particles, which themselves are linked to mean and turbulent flow characteristics \cite{Bialik2012}, a complex dependence of areal deposition rates on flow statistics and bed roughness is expected. These will be presented. 
\bigskip 

\subsection{BEDLOAD ENSEMBLE STATISTICS, A COMPARISON OF EXPERIMENT AND MODELING: }

This project will synthesize the experimental and computational results of projects 3 and 4, in order to explore the validity of the turbulent granular model of bedload ensembles. Fundamentally, this project will check that the mean velocity profiles, energy spectra, and Reynolds stresses obtained in the steady flow PIV experiments can be used to reproduce the ensemble exchange and kinematic ditsributions observed in the steady flow particle tracking experiment using the computational model. Differences between expected and measured results will be highlighted and explained. 
\bigskip 

\subsection{3D LAGRANGIAN BEDLOAD STATISTICS IN UNSTEADY DAM BREAK FLOW:}

There are apparently no studies of ensemble statistics within unsteady flow experiments. This project concerns the analysis of the set of repeated dam break flow experiments already performed. The particle tracking and stereo analysis techniques developed in project 3 will be refined for application to the unsteady flow case. Upon analysis of the 1200 experimental repetitions, ensemble statistics will be developed across the set of realizations. This will lead to time and space varying entrainment, deposition, velocity, and activity distributions, fundamentally expanding the database on statistical characterizations of particle transport in turbulent shear flows. 

\subsection{GRAIN SHAPE AND THE TRANSITION BETWEEN BEDLOAD AND GRANULAR CREEP:}




