\section{INTRODUCTION:}\label{sec:intro}
\linenumbers 

Bedload transport has strong feedback with stream morphology \citep{Church2015,Recking2016}, so its prediction is crucial to a wide range of problems, from aquatic habitat restoration to energy production \citep{McDonald2010, Kondolf2014, Wohl2015a}. Unfortunately, existing capabilities are inadequate. Predictions can deviate by 1 or 2 orders of magnitude from measured values \citep{Gomez1989, Martin2003, Recking2012}. The lack of reliable bedload transport models is a critical research problem. 
\bigskip 

Modelling is challenging because movement is driven by turbulent forces \citep{Schmeeckle2007,Celik2010, Dwivedi2010, Dwivedi2011, Amir2014, Celik2014, Vowinckel2016, Shih2017} and is resisted by a network of variable factors, including granular geometry \citep{Miller1966, Paintal1969, Paintal1971a}, sorting \citep{Parker1982, Lisle1992}, hiding \citep{Egiazaroff1965, Fenton1977}, variations in upstream sediment supply \citep{Madej2009, Elgueta2018}, hydraulic history \citep{Reid1985, Mao2012}, sediment storage \citep{Hoey1992}, channel slope \citep{Prancevic2015, Maurin2018}, channel width \citep{Zimmermann2010}, fine sediment concentration \citep{Wilcock1997}, and collective granular arrangements across a range of spatial scales \citep{Brayshaw1984, Church1998, Hassan2007, Venditti2017}. 
\bigskip 

As a result, sediment fluxes fluctuate widely in time and space, even under apparently steady conditions \citep{Drake1988, Hoey1992, Bohm2004, Radice2009, Singh2009, Turowski2010, Houssais2012, Roseberry2012, Ancey2014, Heyman2016}. Practically, the factors driving and resisting sediment motion appear to be statistical distributions, rather than fixed values, as do the bedload fluxes themeslves. Bedload transport reveals a probability problem \citep{Einstein1937}. In small and medium streams, bedload transport is especially of this character. Large grain sizes relative to channel dimensions, episodic supplies of sediment, and noisy hydrographs all contribute to wide distributions of driving factors, resisting factors, and resultant sediment fluxes \citep{Hassan2005, Church2006, Comiti2012, Church2015}. 
\bigskip 

Since stochastic approaches accomodate distributions, they may be well suited to predicting the flux in these streams. One only needs to specify the statistics of particle exchange and kinematics. Exchange statistics include the areal entrainment rate, areal deposition rate, and the particle diffusivity. Kinematic statistics include probability distributions of velocity, travel time, and hop distance \citep{Ancey2014, Furbish2016, Heyman2016}. If exchange and kinematic statistics are specified, then the statistically expected bedload flux is too, complete with temporal and spatial variations. This prediction would provide ecology and engineering practitioners with a crucially needed foundation. 
\bigskip 

Unfortunately, a practical method to obtain the kinematic and exchange statistics is lacking, so that stochastic models are not useful to applications. The development of a mechanistic linkage to these statistics from measureable channel properties is exposed as a critical research problem. This mechanical linkage can be called closure for the analogous problem in turbulence theory \citep{Heyman2016}. The development of closure is the directive of this research program. The chosen approaches are computational physics and flume experiments. The targets are: (1) a computational model of kinematic and exchange statistics which uses measureable fluid and granular characteristics as input; and (2) 3D observations of kinematic and exchange statistics in steady and unsteady flows, in conjunction with measurements of underlying fluid and granular characteristics. 
\bigskip 

The computational model will include grain-scale resolution and turbulent forcing, permitting exploration of kinematic and exchange statistics over the parameter space of turbulent fluid and granular characteristics. Experiments will develop a progressive dataset of Lagrangian bedload transport of crucial significance to subsequent studies of stochastic transport theory, and experiments will validate the computational model. In the following, the research problem of determining mechanistic closure is given context in \ref{sec:background}, and proposed computational and experimental methods are discussed in  \S \ref{sec:compmethods} and \S \ref{sec:exptmethods} respectively. The ten specific projects composing the research program are introduced in \S \ref{sec:program}, before a summary and some discussion of the greater scope of the research program in \S \ref{sec:conclusion}.

