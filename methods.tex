\section{METHODS:} \label{sec:methods}

\subsection{COMPUTATIONAL METHODS:} \label{sec:compmethods}

The proposed SST-DEM model will express sediment entrainment, motion, and disentrainment with resolution of individual grains. This model will contain grain-grain and fluid-grain interactions. Its inputs will be the prescribed mean velocity profile, turbulent spectrum, Reynolds stress tensor, channel geometry, and the particle size distribution. Its outputs will be large ensembles of Lagrangian trajectories. The model will calculate ensemble kinematic and exchange distributions required by stochastic transport theories in terms of measurable input fluid and granular characteristics. Only one coupled SST-DEM model exists in the literature \citep{Nikora2001a}, although DEM has been employed in many studies, and simple spectral turbulence generators have also featured in some \citep[e.g.][]{Bialik2012}. The SST and DEM methods will now be introduced and their strengths and weaknesses discussed.  
\bigskip 


SST generation was first developed by \citep{Kraichnan1970} to study particle diffusion through a turbulent velocity field. SST has been researched intensely since its development because it generates inflow boundary conditions for numerical simulations required by aerodynamics and heat transfer applications \citep{Shur2014, Dhamankar2015, Wu2017}. The Kraichnan generator develops time independent velocity fields with a prescribed spectral energy density. The generated field is isotropic and homogeneous; its turbulent characteristics are the same in every direction and within any chosen control volume \citep{Wu2017}. 
\bigskip 

Notably, channel flow is anisotropic and inhomogeneous; turbulent intensities vary with distance from the channel wall, and are different in longitudinal, vertical, and transverse directions \citep{Nikora2000}. The SST generators applied in sediment transport literature rescaled isotropic/homogenous velocity fields with empirical turbulent intensities and mean velocity profiles, and resampled the spectral fields repeatedly, until realistically correlated velocity fields were obtained by trial and error \citep{Nikora2001a,Bialik2010}. There is no reason to expect velocity fields generated in this way will exhibit the coherent turbulence  associated with channel flows \citep{Kline1967, Ninto1996, Adrian2007} and considered crucially linked to bedload motion \citep{Vowinckel2016, Shih2017}. In light of the considerable research performed in SST generation, improving upon these earlier approaches to generate SST with these coherent structures should be a matter of borrowing from the extensive literature. 
\bigskip 

Refinements of the Kraichnan method have provided anisotropic inhomogenous SST, permitting the generation of legitimate channel flow with coherent structures and appropriate correlations, provided Reynold stresses are prescribed in addition to energy spectra \citep{Clark2011,Dhamankar2015,Smirnov2001,Yu2014, Shur2014, Adler2018}. The proposed SST-DEM model will incorporate one of these refined generators. In all Kraichnan-like methods, the spectral energy density $E(k)$ is prescribed to a random velocity field $\textbf{u}(\textbf{x},t)$ by developing it as a Fourier series with random phases \citep{Bechara1994,Bailly1999}: \be \textbf{u}(\textbf{x},t) = \sum_\textbf{k} \sqrt{2E(k)\Delta k}\big[ \hat{\textbf{A}}_\textbf{k} \cos({\textbf{k}\cdot \textbf{x}-\omega_\textbf{k}t}) + \hat{\textbf{B}}_\textbf{k}\sin({\textbf{k}\cdot \textbf{x}-\omega_\textbf{k}t})\big].\label{eq:fourier}\ee
The magnitude $|\textbf{k}|$ is distributed across a range related to the minimum and maximum size of eddies in the flow: $k_{min}\leq|\textbf{k}|\leq k_{max}$, $k_{min} \propto 1/l_{max}$, and $k_{max} \propto 1/l_{min}$. The directions of $\textbf{k}$ are randomly selected from a unit sphere. The frequencies $\omega_\textbf{k}$ reflect the time correlations in the field. The unit vectors $\hat{\textbf{A}}_\textbf{k}$ and $\hat{\textbf{B}}_\textbf{k}$ have random directions satisfying $\hat{\textbf{A}}_\textbf{k}\cdot \textbf{k} = 0$ and $\hat{\textbf{B}}_\textbf{k}\cdot \textbf{k} = 0$ so that the field is incompressible. $\Delta k$ is the spacing between adjacent modes. Careful discussion of these topics is available in \cite{Dhamankar2015} and \cite{Wu2017}; to this point, there is little deviation from the original formulation of \cite{Kraichnan1970}.  
\bigskip 

In some generalized Kraichnan generators, the Reynolds stresses are incorporated into the velocity field through a Cholesky decomposition, originally introduced by \cite{Kaiser1962} to study population correlations in social sciences. The stress tensor $\textbf{R}(\textbf{x})$ is decomposed into a product of lower triangular matrices as $\textbf{R} = \textbf{L}\textbf{L}^\text{T}$. The product of $\textbf{L}$ with $\textbf{u}$ provides a turbulent velocity field with prescribed mean velocity profile, turbulent intensities, and spectral energy content \citep{Shur2014,Dhamankar2015,Wu2017}:
\be \textbf{u}'(\textbf{x},t) = \textbf{L}(\textbf{x})\textbf{u}(\textbf{x},t).\ee
The prescribed flow statistics are well represented by $\textbf{u}'$ when several thousand modes $\textbf{k}$ are included in equation \ref{eq:fourier}. In the proposed model, this velocity field will be coupled to individual grains using the classic results for fluid-grain forces \citep[e.g.][]{Maxey1983}. Grains will respond to these synthetic turbulent fluid forces, in addition to grain-grain forces, through a DEM framework. 
\bigskip 

Within DEM, every two grains in contact interact with repulsive, viscous, and frictional forces in tangential and normal directions \citep{Cundall1979,Dziugys2001}. The granular dynamics are governed by the numerical integration of the joint equations of motion with a small timestep. There are powerful codes available which perform these computations with uniform spheres \citep{Plimpton1995} or arbitrary convex polyhedra \citep{Wachs2012}. The computational model in this proposal is just a matter of modifying one of these to include turbulent forcing from the synthetic turbulence. Notably, Dr. Wachs is at UBC and has offered the Grains3D codes, able to handle particles of arbitrary shape, for this simulation \citep{Wachs2012}. Additionally, the author has the ability to generate randomly shaped grains from prescribed statistical shape descriptors, having copied the limited literature on this topic \citep{Mollon2013}; studying the effects of particle shape on entrainment is considered as a tertiary investigation.   
\bigskip 

Admittedly, DEM has a somewhat high computational cost. However, the bedload transport regime exists under flow conditions that do not deviate too far from threshold, and under these flow conditions, there is a well-defined dynamic active layer near the bed surface below which particles do not move \citep{Church2017}. Therefore, grains below this active layer can be locked out of the simulation, providing an appreciable reduction in computational cost. 
\bigskip 

Admittedly, an unavoidable shortcoming of the proposed SST-DEM model is that moving grains influence flow characteristics in their locality \citep{Ferreira2009, Singh2010, Santos2014, Liu2016}, since they provide an additional energy and momentum sink. The SST-DEM model neglects this coupling as the turbulent flow is only described by its statistics. As a result, the validity of spectral turbulent forcing is undermined when particle activities are large. This is not a major issue, since relatively small bedload activities will be sufficient for all of the specific computational investigations outlined in \S \ref{sec:program}, however it is a limitation to keep in mind. In all, the computational model will be, by design, a progressive and appropriate tool to study mechanical closure between measureable channel properties and the ensemble statistics required by stochastic transport models with new depth. 
\bigskip 

\subsection{EXPERIMENTAL METHODS:} \label{sec:exptmethods}
A set of Lagrangian tracking experiments has already been conducted, and the analysis and publication of the data is an important portion of this proposal. Bedload transport of 5mm glass beads was recorded for 20 to 30 minutes across 4 different steady flow conditions, and unsteady dam break flow conditions, including over 1200 repeats, using two high speed (190fps) cameras arranged to achieve 3D binocular vision \citep{Wohler2009}. In a parallel set of experiments, particle image velociometry (PIV) \citep{Raffel1998} was employed to obtain two-dimensional (2D) mean velocity profiles and turbulent statistics for the 4 steady flow conditions. 
\bigskip 

Experiments were designed to yield large Gibbs ensembles characterizing the 3D exchange and kinematic statistics of bedload transport, in conjunction with the statistical properties of the fluid flow. In all, the experiments are a fundamental and progressive contribution to the existing database on Lagrangian transport; they will be the first 3D Lagrangian tracking measurements reported, and the first unsteady flow Lagrangian particle tracking measurements, to the authors knowledge. Additionally, they provide the outputs and inputs of the SST-DEM model, so it can be tested. However, obtaining the experimental ensembles will require delicate particle tracking and stereo reconstruction analyses.
\bigskip 

Particle tracking and stereo reconstruction approaches are now proposed. First, the particle tracking process is divided into object detection and linking stages. Object detection is the identification of particles within each video frame. Linking is the process of connecting particle identities between frames-- a mathematical assignment problem. For object detection, the method of median background subtraction and connected region identification was successfully utilized in previous studies \citep{Radice2006, Heyman2016}, so this will be considered first. 
\bigskip 

If median background subtraction provides insufficient resolution, the second approach will apply a convolutional neural network over individual frames to learn a compressed representation of particle positions. This representation can then be inverted for high fidelity particle positions. This approach has been recently applied very successfully to detect biological cells in microscope images \citep{Xue2017}. Linking detected particles between frames will then follow the Jonker-Volgenant algorithm \citep{Jonker1987}, successfully applied by \cite{Heyman2016} to obtain 2D trajectories. After these steps are complete, collections of 2D trajectories will be obtained, one collection from each camera. 
\bigskip 

Stereo reconstruction proceeds from these stereopairs of 2D trajectories, using disparity relations developed from complementary calibration images \citep{Wohler2009}. These images were obtained with a specially constructed black and white PVC grid in many different orientations within the experimental domain. With these disparity relations, image coordinates of an object taken by one camera can then be mapped to a prediction of its coordinates in the other camera. Additionally, the stereo pair of image coordinates of an object defines its 3D position. 
\bigskip 

To reconstruct 3D trajectories from the stereopairs of 2D trajectories, the set of coordinates of one frame will be mapped onto a prediction of its coordinates in the other frame with the disparity map, and vice versa. Then, linking across cameras is a second assignment problem to be solved with the Jonker-Volgenant algorithm \citep{Jonker1987}. From this analysis, large ensembles of 3D Lagrangian trajectories will result, providing exchange and kinematic statistics across a range of steady and unsteady flow conditions. The number of individual trajectories observed at each flow condition is estimated to be between 15,000 and 100,000: the upper end of this estimated range is much larger than all other studies of this type \citep[e.g.][]{Roseberry2012, Heyman2016}. 
\bigskip 
