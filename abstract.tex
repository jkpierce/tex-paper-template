\begin{abstract}
\noindent A three year computational and experimental research program of ten specific projects is proposed to investigate the kinematic and exchange statistics at the foundation of contemporary stochastic bedload transport theory. Computations will drive the Lagrangian transport of a discrete granular phase with a synthetic turbulent flow. Entrainment, motion, and deposition will result. The flow and granular characteristics will be prescribed to measureable channel properties. Virtual trajectories will be concatenated into ensembles, simulating kinematic and exchange statistics. This is an investigation of the deterministic mechanics underlying stochastic transport. Experiments will develop three dimensional Lagrangian transport ensembles with binocular computer vision. Mean and turbulent flow properties will be measured with particle image velociometry. Steady and unsteady flow conditions will be considered. The exchange and kinematic statistics obtained will be the first ever reported in 3D and unsteady flows, with promise to influence the developing stochastic theory of bedload transport, and they will test the computational model. The directive is to develop understanding of the mechanistic underpinnings of stochastic theories, to work toward rendering them applicable to engineering and ecological problems. 






















\end{abstract}

\begin{comment}


Three steady flow conditions and one unsteady flow condition have already been measured. The unsteady flow was generated by a dam break; to obtain an ensemble, it was repeated 1200 times. 


Experiments will develop these statistics from large ensembles of three dimensional (3D) bedload trajectories obtained with binocular computer vision, alongside mean fluid velocity profiles and turbulent statistics from particle image velociometry. Steady and unsteady flow conditions are considered. 


will generate these statistics from measureable channel properties. Synthetic turbulence will drive a virtual granular phase represented by discrete elements. Entrainment, motion, and deposition will result. Virtual trajectories will be concatenated into ensembles, developing kinematic and exchange statistics. 




Thereby generalizing earlier efforts by extending them to 3D and unsteady flow conditions. Experiments also provide all inputs and outputs of the computational model, so it can develop with this dataset. 

Experiments over three steady flow conditions and one unsteady flow condition comprised of over 1000 repeated trials of a dam break flow have already been conducted. These experiments will be a fundamentally significant contribution to the research community, surpassing earlier approaches by extending measurements to 3D and considering the effects of unsteady flow on kinematic and exchange statistics. Additionally, the experiments will provide all inputs and outputs of the computational model, so it can be tested. 

Computations will develop synthetic turbulence from input statistics characterizing granular and fluid phases. Turbulent forcing will drive a virtual granular phase represented by discrete elements. 
The ensemble statistics obtained will be the first of their kind, enriching the small number of earlier approaches by considering 3D trajectories and unsteady flow conditions.

Kinematic and exchange statistics are the foundation of contemporary transport theories, but they are poorly understood.  Synthetic turbulent forcing will drive the entrainment, motion, and deposition of discrete grains in the computational scheme. The turbulent and granular properties will be fixed by experimentVirtual trajectories will be concatenated into ensembles, developing the required kinematic and exchange statistics. 







The target is a computational model predicting Lagrangian exchange and kinematic statistics from measureable channel properties. Particle tracking experiments will provide a progressive dataset of exchange and kinematic statistics to the research community and test the developed model. These statistics are required for application of stochastic transport theories to engineering and ecological problems.



Therefore, computations will constrain a missing link between stochastic transport theories and applications  



Therefore, computations will map measured fluid and granular statistics to the ensemble statistics of particle transport. The model will be validated with experimental data. he relevance of the proposed work to these problems and to ecological and engineering practice is discussed. 
\end{comment}